% LaTeX Curriculum Vitae Template
%
% Copyright (C) 2004-2008 Jason Blevins <jrblevin@sdf.lonestar.org 
% http://jblevins.org/projects/cv-template
%
% You may use use this document as a template to create your own CV
% and you may redistribute the source code freely. No attribution is
% required in any resulting documents. I do ask that you please leave
% this notice and the above URL in the source code if you choose to
% redistribute this file.

\documentclass[a4paper]{article}

\usepackage{hyperref}
\usepackage{geometry}
%\usepackage{lmodern}
\usepackage[osf]{mathpazo}
\usepackage[utf8]{inputenc}
\usepackage{fancyhdr}
\usepackage{pdfpages}
\usepackage{natbib}


%\hypersetup{
%  colorlinks,
%  urlcolor = black,
%  pdfauthor={Jason R. Blevins},
%  pdfkeywords={economics empirical industrial organization econometrics
%    microeconomics applied microeconomics},
%  pdftitle={Curriculum Vitae},
%  pdfsubject={Curriculum Vitae},
%  pdfpagemode=UseNone
%}

\geometry{textheight=9.5in, textwidth=6.5in}

% Customize section headings
\usepackage{sectsty}
\sectionfont{\rmfamily\bfseries\large}
\subsectionfont{\rmfamily\mdseries\scshape\normalsize}
\subsubsectionfont{\rmfamily\bfseries\upshape\normalsize}

% Don't indent paragraphs.
\setlength\parindent{0em}
\setlength\parskip{0.5em}

% Make lists without bullets
\renewenvironment{itemize}{
  \begin{list}{}
    { \setlength{\itemsep}{5pt}
      \setlength{\parsep}{0pt}
      \setlength{\topsep}{0pt}
      \setlength{\leftmargin}{0em} } }{
  \end{list}}

\pagestyle{fancy}

\begin{document}
	
	
			%\vspace{-3mm}
			
			%\vspace{-2mm}
			%\nocite{vejdemo_two_2015,vejdemo_semantic_2016,vejdemo_cross-linguistic_2010,vejdemo_database_2016,vejdemo_skarp_2007,vejdemo-johansson_comparing_2014,koptjevskaja-tamm_there_2015,vejdemo_extended_2016,vejdemo_triangulating_2017}
			
			
			%\bibliographystyle{plainnat}
			%\bibliography{Vejdemo-180313} 
	
		\lhead{\textbf{List of publications: Susanne Vejdemo}} 
			\subsection*{\textbf{Peer-reviewed original articles}}	
					
				
					\textbf{Susanne Vejdemo} and Thomas Hörberg.
					\newblock 2016.
					\newblock Semantic factors predict the rate of lexical replacement of content
					  words.
					\newblock In \emph{PLoS ONE 11(1): e0147924}.
					
					
						\textbf{Susanne Vejdemo}, Carsten Levisen, Thorhalla~G. Beck, Cornelia von Scherpenberg,
						  Åshild Næss, Martina Zimmerman, Linnaea Stockall, and Matthew Whelpton.
						  \newblock 2015.
						\newblock Two kinds of pink: Development and difference in germanic colour
						  semantics.
						  \newblock In \emph{Language Sciences. 49:\penalty0 19--34. 10.1016/j.langsci.2014.07.007.}
						  
						  	
						  	Mikael Vejdemo-Johansson, \textbf{Susanne Vejdemo}, and Carl-Henrik Ek.
						  	\newblock 2014.
						  	\newblock Comparing distributions of color words: Pitfalls and metric choices.
						  	\newblock In \emph{PLoS ONE  9\penalty0 (2):\penalty0 e89184}.
						  	
	

		
		\subsection*{\textbf{Peer-reviewed conference contributions}}
			\textbf{Susanne Vejdemo}.
			\newblock 2016.
			\newblock To database meaning: Building the typological database of temperature terms.
			\newblock In \emph{Annual Meeting of the Michigan Linguistic Society, the University of Michigan}
			
			
				
					\textbf{Susanne Vejdemo}.
					\newblock 2010.
					\newblock Cross-linguistic lexical change: Why, how and how fast?
					\newblock In \emph{Proceedings of {WIGL} 2010}, volume~8 of \emph{{LSO} Working  Papers in Linguistics}.
			
		
		\subsection*{\textbf{Peer-reviewed books and book chapters}}
	
			\textbf{Susanne Vejdemo}.
				\newblock 2017.
					\newblock \emph{Triangulating Perspectives on Lexical Replacement: From
					  Predictive Statistical Models to Descriptive Color Linguistics.}
					\newblock PhD thesis, Stockholm University, Stockholm.
	
		
	\textbf{Susanne Vejdemo} and Sigi Vandewinkel.
	\newblock 2016.
		\newblock Extended uses of temperature terms across languages.
		\newblock In Maria Koptjevskaja-Tamm and Päivi Juvonen, editors,
		  \emph{Lexicotypological approaches to semantic shifts and motivation patterns
		  in the lexicon}, pages 249--284. 
		  
		  
		  
		  	
		  	Hunter Lockwood and \textbf{Susanne Vejdemo}.
		  	\newblock 2015.
		  	\newblock “There is no thermostat in the forest” – the ojibwe temperature 	  term system.
		  	\newblock In Maria Koptjevskaja-Tamm, editor, \emph{The Linguistics of
		  	  Temperature}, volume 107 of \emph{Typological Studies in Language}, pages
		  	  721--741. John Benjamins Publishing Company.
	%	  ***********
		
	
		
	
		\subsection*{\textbf{Other publications }}	
	
	
						  		
						  	\textbf{Susanne Vejdemo}.
						  	\newblock 2007.
						  		\newblock \textit{Skarp, vass och sharp – semantiska relationer hos tre
						  		  perceptionsadjektiv}.
						  		\newblock Master thesis, University of Stockholm.
	
			
	
		
	\newpage
			\lhead{\textbf{List of publications: Maria Koptjevskaja Tamm}} 
				\subsection*{\textbf{Peer-reviewed original articles}}	
				
				\textbf{Maria Koptjevskaja-Tamm},  Martine Vanhove and Peter Koch.
				\newblock 2007
				\newblock Typological approaches to lexical semantics.
				\newblock In \textit{Linguistic Typology, 11-1: 159 – 186}.
				
		\subsection*{\textbf{Peer-reviewed original articles/Peer-reviewed book chapters}}
	  
	  
	  
	  	   \textbf{Maria Koptjevskaja-Tamm} and Magnus Sahlgren.
	  	   \newblock 2014.
	  	   \newblock Temperature in the Word Space: sense exploration of temperature expressions using word-space modeling. 
	  	   \newblock In Szmrecsanyi, B. and B. Wälchli (eds),
	  	  \textit{Linguistic variation in text and speech, within
	  	   and across languages} 	  (Series: Linguae et Litterae: Publications of the School of Language
	  	   and Literature, Freiburg Institute for Advanced Studies). Berlin: de Gruyter, 231 – 267.
	  
	  	
	  	   \textbf{Maria Koptjevskaja-Tamm}, Dagmar Divjak and Ekaterina Rakhilina.
	  	  \newblock 2010.
	  	  \newblock Aquamotion verbs in Slavic and Germanic: A case study in lexical typology. 
	  	  \newblock In Driagina-Hasko, V. and R. Perelmutter (eds.), \textit{New approaches to Slavic verbs of motion}. Amsterdam: Benjamins, 315-342.
	  	  
	  	  	   
	  	  	   	   \textbf{Maria Koptjevskaja-Tamm}.
	  	  	  \newblock 2009. 
	  	  	  \newblock “A lot of grammar with a good portion of lexicon”: towards a typology of partitive and pseudo-partitive nominal constructions. 
	  	  	  \newblock In Helmbrecht, J., N. Yoko, S. Yong-Min, S. Skopeteas and E.Verhoeven (eds.),\textit{ Form and Function in Language Research}. Berlin: Mouton de Gruyter, 329 – 346.
	  	  
	  	  
	  	
	  
	   \textbf{Maria Koptjevskaja-Tamm}.
	    \newblock  2008.
	    \newblock Approaching   lexical typology.
	    \newblock In Vanhove, M. (ed.), \textit{From polysemy to semantic change: a typology of lexical semantic associations}. Amsterdam: Benjamins, 3–52.
	    

	
	
	 
	 
	  \subsection*{\textbf{Peer-reviewed edited volumes}}
	 
	
		   Päivi Juvonen  and \textbf{Maria Koptjevskaja-Tamm} (eds.).
		  \newblock 2016.
		  \newblock \textit{The lexical typology of semantic shifts}.
		 \newblock  Berlin – New York: de Gruyter Mouton.
		 
		 
		 	  \textbf{Maria Koptjevskaja-Tamm} (ed.)
		 	  \newblock 2015
		 	  \newblock \textit{The linguistics of temperature. }
		 	\newblock	   Amsterdam: John Benjamins.
	
	  \textbf{Maria Koptjevskaja-Tamm} and Martine Vanhove (eds.).
	  \newblock 2012
	  \newblock New directions in
	   lexical typology. 
	   \newblock In \textit{A special issue of Linguistics,
	   50–3: 373–743}.


\subsection*{\textbf{ Peer-reviewed book chapters / Research review articles }}


	    \textbf{Maria Koptjevskaja-Tamm} and Henrik Liljegren.
	    \newblock 2017.
	    \newblock \textit{Lexical semantics and areal linguistics}. \newblock In Hickey, R. (ed.),
	  The Cambridge Handbook of Areal Linguistics.
	  Cambridge: Cambridge University Press, 204 – 236.

	  \textbf{Maria Koptjevskaja-Tamm}.
	  \newblock 2015.
	  \newblock \textit{Semantic typology}.
	   \newblock In Dabrowska, E. and D. Divjak (eds.),  Handbook of Cognitive Linguistics, 453 – 472.
	  Handbooks of Linguistics and Communication Sciences (HSK), 39, Berlin – New York: de Gruyter Mouton.
	

	
	\newpage

	
	\includepdf[pages=-]{publications-Johansson.pdf}
	


\end{document}
